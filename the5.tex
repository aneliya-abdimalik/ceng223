\documentclass[12pt]{article}
\usepackage[utf8]{inputenc}
\usepackage{float}
\usepackage{amsmath}


\usepackage[hmargin=3cm,vmargin=6.0cm]{geometry}
%\topmargin=0cm
\topmargin=-2cm
\addtolength{\textheight}{6.5cm}
\addtolength{\textwidth}{2.0cm}
%\setlength{\leftmargin}{-5cm}
\setlength{\oddsidemargin}{0.0cm}
\setlength{\evensidemargin}{0.0cm}

%misc libraries goes here



\begin{document}

\section*{Student Information } 
%Write your full name and id number between the colon and newline
%Put one empty space character after colon and before newline
Full Name : Aneliya Abdimalik
Id Number :  2547651

% Write your answers below the section tags
\section*{Answer 1}
Lets first consider an adjacency matrix of this graph to answer following questions easier.
$$
\begin{bmatrix}
&a&b&c&d&e\\
a&0&1&1&0&1\\
b&1&0&1&0&1\\
c&1&1&0&1&0\\
d&0&0&1&0&1\\
e&1&1&0&1&0\\
\end{bmatrix}
$$
\subsection*{a)}
Node a has 3 1s, therefore degree is 3.\\
Node b has 3 1s, therefore degree is 3.\\
Node c has 3 1s, therefore degree is 3.\\
Node d has 2 1s, therefore degree is 2.\\
Node e has 3 1s, therefore degree is 3.\\
\\
Sum of all degrees is 3+3+3+2+3=14.\\
\subsection*{b)}
Total number of non-zero entries is 14.
\subsection*{c)}
Following Incidence matrix:
$$
\begin{bmatrix}
&1&2&3&4&5&6&7\\
a&1&1&1&0&0&0&0\\
b&0&1&1&0&1&0&0\\
c&1&1&0&1&0&0&0\\
d&0&0&1&0&0&1&0\\
e&1&1&0&0&1&0&0\\
\end{bmatrix}
$$
Total number of zero entries is 15.\\
\subsection*{d)}
Graph G doesn't have a complete graph with at least 4 vertices as a subgraph because all subgraphs of G have 3 verices. \\
\subsection*{e)}
No, it isn't bipartite because node a and b are both connected to node e which failes the condition of bipartite.\\
\subsection*{f)}
edges=7, so $2^7=128$\\
\subsection*{g)}
Simple longest path is length of 4 edges. For example,a-b-c-d-e or b-a-c-d-e.\\
\subsection*{h)}


\subsection*{i)}
\subsection*{j)}
\subsection*{k)}
\subsection*{l)}

\section*{Answer 2}


\section*{Answer 3}


\section*{Answer 4}


\end{document}